
\chapter{学位论文规范}

\section{学位论文的纸张、字体、用纸及顺序、页码安排}
\subsection{论文纸张、行距、页边距及页眉}
论文中每章、节的格式和版面要求整齐划一、层次清楚。

论文用纸一律为A4纸张,上方、下方、左侧页边距为25毫米,右侧页边距为20毫米,装订线为4毫米。

页眉从正文开始到最后。奇数页的页眉标明章的名称,偶数页的页眉标“西安石油大学硕士(或博士)学位论文”字样。页眉位于每一页的最上方,用5号宋体,居中排列,页眉之下可划1条贯穿页面的直线,例:
\begin{table}[H]
    \centering
	\begin{tabular}{p{\textwidth}<{\centering} }
        \zihao{5}西安石油大学博士学位论文\\ \hline
	\end{tabular}
\end{table}
\noindent{或:}
\begin{table}[H]
    \centering
	\begin{tabular}{p{\textwidth}<{\centering} }
        \zihao{5}西安石油大学硕士学位论文\\ \hline
	\end{tabular}
\end{table}

\subsection{论文用字与打印、页码安排}
论文一律使用简化汉字,要求打印清楚,少量中、英文无法打印的文字符号允许手写,但须清晰整洁。

正文中文及摘要用字为小4号宋体,英文为“Times News Roman”,行距采用20磅的行距,不得使用不合规定的简化字、复合异体字或乱造汉字。

页码为宋体小5号,从论文正文的第一页开始,在每页的底端居中排列。要求论文双面复印,正文中
不允许有空白页。论文正文前的创新性说明及学位论文使用授权的说明、中文摘要、英文摘要、目录、主要符号表等页以Ⅰ、Ⅱ、Ⅲ、Ⅳ……等页码另排,页眉为“中文摘要”、“英文摘要”、“目录”等,并单页打印。\textbf{(彩色插图、表格等可单页打印,除此以外均为双面打印)}

\subsection{论文顺序}
论文顺序依次为:封面、学位论文创新性说明及学位论文使用授权的说明、中文摘要、英文摘要、目录、主要符号表、正文、结论、致谢、参考文献、附录、攻读学位期间发表的论文、专利、获奖等。主要符号表和附录可按需列入。

\section{学位论文的封面、创新性说明及学位论文使用授权的说明、中文摘要、英文摘要、目录及主要符号表的规范}
\subsection{封面}
论文封面按照研究生部统一要求的格式打印(见《硕士学位论文封面》)。论文题目字数一般应在25字以内(应与开题报告完全一致)。
\subsection{学位论文创新性说明及学位论文使用授权的说明}
格式见《西安石油大学研究生学位论文写作规范》(2017年9月)附件1:单页打印。
\subsection{中文摘要}
中文摘要应在300~500字之间。论文摘要一般包括:论文的目的、方法、结果和结论,提供尽可能多的定性或定量的信息。

论文摘要应突出论文的新见解和创造性成果。用句应精炼概括,并列出论文的关键词3~5个,关键词应从《汉语主题词表》中摘选,当《汉语主题词表》中的词不足以反映主题,可由作者设计关键词,但须加注,加黑表示。

摘要中应注明论文的类型。论文的类型分基础研究、应用研究、开发研究和其它共四种,作者根据自己的研究工作,从中选择一种。

摘要前的题头:在摘要的上方,应写上论文题目、专业、导师和研究生的签名(中文)(见附件2),并加黑表示。摘要的最下方写上(没有则不写):本研究得到***基金的资助。
格式见《西安石油大学研究生学位论文写作规范》(2017年9月)附件2,单页打印。

\subsection{英文摘要}
摘要前需写上题头,要有导师和研究生的签名(英文或汉语拼音)(见附件3), 并加黑表示。以汉语拼音形式签名时,姓在前,名在后,姓和名之间留出一个空格,英文摘要撰写要求如下:

(1)用词应准确,使用本学科通用的词汇;

(2)摘要中主语(作者)常常省略,因而一般使用被动语态。应使用正确的时态并要注意主、谓的一致。必要的冠词不能省略;

(3)关键字(Key words)按相应专业的标准术语写出,并加黑表示;

(4)中、英文摘要的内容须一致。

格式见《西安石油大学研究生学位论文写作规范》(2017年9月)附件3,单页打印。

\subsection{目录}
格式见《西安石油大学研究生学位论文写作规范》(2017年9月)附件4。目录中章、节号均使用阿拉伯数字,如章为一,分层次序为1.1及1.1.1等层阶。单页打印。

\subsection{主要符号表}
单页打印

(1)全文中常用的符号及意义在主要符号表中列出;

(2)符号排列顺序按英文及其它相关文字顺序排出;

(3)主要符号表的页号另编。

\section{学位论文正文的规范}
正文一般由绪论、理论分析、系统仿真、数值计算、实验研究、软件开发、工程实现等几部分组成,可按相对独立的内容分章。全文与参考文献紧密结合,重点论述研究生本人的独立工作内容和创造性见解。论文不得模糊研究生与他人的工作界限,参考或引用了他人的学术成果或学术观点,必须给出参考文献的出处、作者等。严禁抄袭、占有他人的成果。

    正文层次格式如
\begin{center}
              第*章  ******(3号宋体)………(页面居中)
\end{center}

\noindent 1.1 ******(4号,宋体,加黑) ………(不接排,单独一行)

\noindent 1.1.1 ***(4号,宋体,加黑) ………(不接排,单独一行)

\noindent \hspace{1em}a. ******(4号,宋体)  ******(接排)

\subsection{绪论}
(1)须清楚、严谨地论述国内外关于本研究的发展水平与存在的问题;

(2)应明确地论述本论文研究的目的意义;

(3)介绍本文工作的构思和主要工作任务;

(4)介绍课题的来源与背景。

\subsection{理论分析与系统仿真}
(1)理论分析要求思路清晰、立论正确、推导严谨、准确无误;

(2)理论分析中的计算实例及仿真(摸拟),须给出完整的条件;

(3)理论分析中过长的分析与证明,无论是否已经公开发表,均应在学位论文附录中列出。

\subsection{数值计算}
数值计算可以作为相对独立的内容成章,在数值计算中要求:

(1)清楚地说明数值计算的出发点及基本方程,对忽略不计的项须给出令人信服的说明;

(2)清楚地交待离散化方法,使用的计算机机型与编程语言;

(3)清楚地交待计算对象及其物理模型的条件,一般情况下,数值计算须和理论分析、实验研究的对象条件相同或接近,以保证计算有代表意义。

\subsection{实验研究及工程实现}
必须高度重视实验研究。实验研究及工程实现包括内容及注意事项:

(1)实验设计,即通过实验要达到的目的,实验测试的方法和手段;

(2)介绍实验装置,须注明所用仪器、传感器等设备的规格型号、标定情况等;

(3)介绍实验原理与测试方法,应尽可能地配图说明;

(4)实验数据要充分,数据处理要合理;

(5)要注意实验结果的精确性、可靠性和重复性,一定要实事求是,严禁弄虚做假、随意修改实验结果;

(6)实验结果要完整,实验条件须合理并要标注完全,据此要能检查实验结果及理论分析、重要参数等是否正确;

(7)工程实现要有实现方法和成果的完整叙述。

\subsection{结论}
要求简明扼要地概括全部论文所得的若干重要结果,包括理论分析、数值计算及实验研究等结果,着重介绍研究生本人的独立研究和创造性成果及其在本学科领域中的地位和作用。用词要准确、精炼,实事求是。

\subsection{插图、表格、公式}
\noindent \hspace{1em}a.插图

(1)所有插图按分章编号,如第一章的第3张插图为“图1-3”,所有插图均需有图题(图的说明),图号及图题应在图的下方居中标出且采用5号黑体字。

(2)一幅图如有若干幅分图,均应按顺序编排分图号,用(a)(b)(c)……;

(3)插图须紧跟文字表述处。在正文中,一般应先引用图号,后列出插图,特殊情况需延后的插图不应跨节;

(4)图形符号及各种线型画法须按照现行的国家标准;

(5)坐标系中坐标上须注明标度值,并标明坐标轴所表示的物理量名称及量纲,均应按国际标准(SI)标注,例如:kW,m/s,N,m……等;

(6)提供照片的尺寸应大小适宜,主题明确,层次清楚,金相照片一定要有放大倍数;

(7)插图应具有“自明性”,即只看图、图题和图例,不阅读正文,就可理解图意;

(8)插图中须完整标注条件,如实验条件、结构参数等;

(9)图中用字最小为5号字;

(10)使用他人插图须注明出处。

\noindent \hspace{1em}b.表格

(1)表格应按章编号,如“表2-1”,并需有表题,表题字用5号黑体。

(2)表号、表题应从表格上方居中排列;

(3)表格的设计应紧跟文字表述处,若为大表或作为工具使用的表格,可作为附表在附录中给出;

(4)表中各物理量及量纲均按国际标准(SI)及国家规定的法定符号和法定计量单位标注;

(5)使用他人表格须注明出处。

\noindent \hspace{1em}c.公式

(1)公式均需有公式号;

(2)公式号按章编排,如式(2-3),公式号按整个页面右对齐;

(3)公式中各物理量及量纲均按国际标准(SI)及国家规定的法定符号和法定计量单位标注,禁止使用已废弃的符号和计量单位;

(4)公式中用字、符号、字体要符合学科规范。

\subsection{致谢}
(1)致谢中主要感谢导师和对论文工作直接有贡献或帮助的人士和单位。谢辞谦虚诚恳,实事求是。不应过分地感谢与论文无直接关系的人士的帮助,如研究生的家属及亲朋好友等;

(2)致谢中还应感谢提供研究经费及实验装置的基金会或企业等单位和人士;

(3)致谢中一般不用第一人称(可用“作者”),结束时不签研究生姓名。

\subsection{参考文献}
(1)参考文献一般应是作者亲自考察过的对学位论文有参考价值的文献,除特殊情况外,一般不应间接使用参考文献;

(2)参考文献应具有权威性,要注意引用最新的文献;

(3)引用他人的学术观点或学术成果,必须列在参考文献中;

(4)参考文献在整个论文中按出现次序依次列出;

(5)硕士学位论文参考文献一般应在20~50篇,其中国外文献5篇以上;

著录格式见《西安石油大学研究生学位论文写作规范》(2017年9月)附件5。

\section{学位论文的附录,攻读学位期间参加科研情况及获得的学术成果}
\subsection{附录}
附录的内容包括:

(1)正文中过长的公式推导与证明过程可在附录中依次给出;

(2)与本文紧密相关的非作者本人的分析、证明及工具用表格等;

(3)在正文中无法列出的实验数据。

\subsection{攻读学位期间参加科研情况及获得的学术成果}
在学位论文的最后,应附上研究生本人在攻读学位期间所发表的论文,获得的专利、获奖、鉴定及工程实现的社会评价及有关资料(一般只写目录清单即可)。

\vspace{2em}
\noindent\textbf{另附:详细摘要写作要求:}

硕士学位论文详细摘要应具有独立性和自含性,即不阅读学位论文全文,就能获得必要的信息。详细摘要中有数据、有结论,是一篇完整的短文,可以独立使用,可以引用,可以用于工艺推广。详细摘要的内容包含与学位论文同等量的主要信息。详细摘要一般应说明研究工作的目的、研究方法、实验方法、结果和结论等,同时阐明论文的新见解。

硕士学位论文详细摘要一般要求在5000-1万字以内。摘要中不要附图表、图片、公式及计算过程(特殊情况除外)。字体为小4号宋体。

详细摘要中每页的正文、页码、页眉格式要求等同于学位论文正文写作规范;页眉如下所示:
\begin{table}[H]
    \centering
	\begin{tabular}{p{\textwidth}<{\centering} }
        \zihao{5}西安石油大学硕(博)士学位论文详细摘要\\ \hline
	\end{tabular}
\end{table}