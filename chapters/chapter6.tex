\chapter{模板使用方法}
\section{模板包目录结构说明}
\dirtree{%
 .1 /.
 .2 bib\DTcomment{参考文献目录}.
 .3 tex.bib\DTcomment{参考文献文件}.
 .2 chapters\DTcomment{章节文件目录}.
 .3 abstract.tex\DTcomment{摘要}.
 .3 appendix.tex\DTcomment{附录}.
 .3 resume.tex\DTcomment{学术成果}.
 .3 thanks.tex\DTcomment{致谢}.
 .3 chapter*.tex\DTcomment{各章节源文件}.
 .2 figures\DTcomment{论文用图目录}.
 .2 font\DTcomment{字体目录}.
 .3 FZxiaobiao.ttf\DTcomment{需要使用的方正小标宋简体字体}.
 .2 gbt7714-2005.bst\DTcomment{参考文献格式标准使用gbt7714-2005(学校给出的是GB7714-87)}.
 .2 templet.tex\DTcomment{模板主控文件}.
 .2 XSYUthesis.cfg\DTcomment{作者信息配置}.
 .2 XSYUthesis.cls\DTcomment{模板样式设置}.
 .2 XSYUthesis.def\DTcomment{论文信息配置}.
 }
\section{环境搭建}
本模板是在Windows 10(2004)下,安装TexLive2020及Visual Studio Code后,修改制作完成的。
\subsection{TexLive的安装}
\begin{enumerate}
    \item 下载TexLive前往http://mirror.ctan.org/systems/texlive/tlnet/install-tl-windows.exe。
    \item 安装过程参考https://blog.csdn.net/weixin\_39892850/article/details/105468247。
\end{enumerate}
\subsection{Visual Studio Code的安装}
\begin{enumerate}
    \item 安装过程参考https://zhuanlan.zhihu.com/p/106357123。
    \item vscode的Latex环境配置参考https://zhuanlan.zhihu.com/p/38178015。
\end{enumerate}
\subsection{字体的安装}
我校研究生学位论文封面中使用了方正小标宋。windows系统下需要自行安装。请谨记:非必要请勿移动fornt/FZxiaobiao.ttf的文件位置,如需移动,请将 XSYUthesis.cls 文件中的命令\verb|\|setCJKfamilyfont\{xbsong\}[AutoFakeBold=true,Path=./font/]\{FZxiaobiao.ttf\}中的Path设置为移动后的路径。
\subsection{其他问题}
\begin{itemize}
    \item 正文中未见参考文献列表及参考文献为[?]的情况:请在使用xelatex命令之前,多使用bibtex命令编译几遍。
\end{itemize}
