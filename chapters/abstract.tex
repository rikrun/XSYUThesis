\begin{abstract}
\setlength\parindent{2em}
摘要是学位论文的内容不加注释和评论的简短陈述,简明扼要陈述学位论文的研究目的、内容、方法、成果和结论,重点突出学位论文的创造性成果和观点。摘要包括中文摘要和英文摘要,硕士学位论文中文摘要字数一般为~300~~~500~字左右,博士学位论文中文摘要字数一般为~1000~~~1500~字左右。英文摘要内容与中文摘要内容保持一致,翻译力求简明精准。摘要的正文下方需注明论文的关键词,关键词一般为~3~~~5~个,关键词和关键词之间用逗号并空一格。\\
\indent 摘要第二段。\\
\indent 摘要第三段。\\
\keywords{XXX,\quad{}XXX,\quad{}XXX,\quad{}XXX,\quad{}XXX} \\
\end{abstract}

\begin{englishabstract}
    \setlength\parindent{2em}
    ~~~~~~~~The Abstract is a brief description of a thesis or dissertation without notes or comments. It represents concisely the research purpose, content, method, result and conclusion of the thesis or dissertation with emphasis on its innovative findings and perspectives. The Abstract Part consists of both the Chinese abstract and the English abstract. The Chinese abstract should have the length of approximately 1000 Chinese characters for a master thesis and 1500 for a Ph.D. dissertation. The English abstract should be consistent with the Chinese one in content. The keywords of a thesis or dissertation should be listed below the main body of the abstract, separated by commas and a space. The number of the keywords is typically 3 to 5.\\
    \indent Second pargraph\\
    \indent Third pargraph\\
\englishkeywords{XXX,\space{}XXX,\space{}XXX,\space{}XXX,\space{}XXX} \\

\end{englishabstract}


\begin{symbollist}
\item \makebox[8em][l]{~}\makebox[16em][l]{符号} \makebox[8em][l] {符号名称}
\item \makebox[8em][l]{~}\makebox[16.2em][l]{$\in$}\makebox[8em][l] {属于}
\item \makebox[8em][l]{~}\makebox[16.2em][l]{$\mathbb{R}$}\makebox[8em][l] {实数集}
\item \makebox[8em][l]{~}\makebox[16em][l]{$w$} \makebox[8em][l] {权重}
\item \makebox[8em][l]{~}\makebox[16em][l]{$x$} \makebox[8em][l] {样本}
\item \makebox[8em][l]{~}\makebox[16em][l]{$y$} \makebox[8em][l] {标签}
\item \makebox[8em][l]{~}\makebox[16em][l]{$M$} \makebox[8em][l] {特征维数}
\item \makebox[8em][l]{~}\makebox[16em][l]{$N$} \makebox[8em][l] {样本数量}
\item \makebox[8em][l]{~}\makebox[16em][l]{$\eta$} \makebox[8em][l] {学习率}
\item \makebox[8em][l]{~}\makebox[16em][l]{$\mathcal{F}^{-1}$} \makebox[8em][l] {逆傅里叶变换}
\item \makebox[8em][l]{~}\makebox[16em][l]{$\gamma$} \makebox[8em][l] {弱分类器更新率}
\end{symbollist}

\begin{abbreviationlist}
\item \makebox[8em][l]{缩略语}  \makebox[16em][l]{英文全称}  \makebox[16em][l]{中文对照}
\item \makebox[4em][l]{SVM}    \makebox[20em][l]{Support Vector Machine}    \makebox[16em][l]{支持向量机}
\item \makebox[4em][l]{EM}    \makebox[20em][l]{expectation–maximization}    \makebox[16em][l]{最大期望}
\item \makebox[4em][l]{WTS}   \makebox[20em][l]{Weighted Tensor Subspace}    \makebox[16em][l]{加权张量子空间}
\item \makebox[4em][l]{PCA}    \makebox[20em][l]{Principal Component Analysis}    \makebox[16em][l]{主成分分析}
\item \makebox[4em][l]{IPCA}    \makebox[20em][l]{Incremental PCA}    \makebox[16em][l]{增量主成分分析}
\item \makebox[4em][l]{HOG}    \makebox[20em][l]{Histogram of Oriented Gradient}    \makebox[16em][l]{方向梯度直方图}
\item \makebox[4em][l]{2D-LDA}    \makebox[20em][l]{2D Fisher Linear Discriminant Analysis}  \makebox[16em][l]{二维Fisher线性判别分析}
\item \makebox[4em][l]{AVT}    \makebox[20em][l]{Attentional Visual Tracking}    \makebox[16em][l]{注意视觉跟踪}
\item \makebox[4em][l]{RF}    \makebox[20em][l]{Random Forest}    \makebox[16em][l]{随机森林}
\item \makebox[4em][l]{FFT}    \makebox[20em][l]{Fast Fourier Transformation}    \makebox[16em][l]{快速傅里叶变换}
\item \makebox[4em][l]{MOSSE}   \makebox[20em][l]{Minimum Output Sum of Squared Error filter}    \makebox[10em][l]{最小平方误差滤波器}
\item \makebox[4em][l]{CFT}    \makebox[20em][l]{Correlation Filter Tracker}    \makebox[16em][l]{相关滤波跟踪器}
\item \makebox[4em][l]{DFT}    \makebox[20em][l]{Discrete Fourier Transform}    \makebox[16em][l]{离散傅里叶变换}
\item \makebox[4em][l]{KCF}    \makebox[20em][l]{Kernelized Correlation Filter}    \makebox[16em][l]{核相关滤波器}
\item \makebox[4em][l]{CLE}    \makebox[20em][l]{Center Location Error}    \makebox[16em][l]{中心位置误差}
\item \makebox[4em][l]{OP}    \makebox[20em][l]{Overlap Precision}    \makebox[16em][l]{重叠精度}
\item \makebox[4em][l]{DP}    \makebox[20em][l]{Distance Precision}    \makebox[16em][l]{距离精度}
\item \makebox[4em][l]{ASMM}    \makebox[20em][l]{Atkinson–Shiffrin Memory Model}    \makebox[16em][l]{AtkinsonShiffrin 内存模型}
\item \makebox[4em][l]{MUSTer}    \makebox[20em][l]{MUlti-Store Tracker}    \makebox[16em][l]{多贮存跟踪器}
\item \makebox[4em][l]{KNN}    \makebox[20em][l]{K-Nearest Neighbor}    \makebox[16em][l]{K-最近邻}
\item \makebox[4em][l]{HOG}    \makebox[20em][l]{Histogram of Oriented Gradient}    \makebox[16em][l]{方向梯度直方图}
\item \makebox[4em][l]{ALM}    \makebox[20em][l]{Augmented Lagrange Method}    \makebox[16em][l]{增强拉格朗日方法}
\item \makebox[4em][l]{ADMM}    \makebox[20em][l]{Alternating Direction Method of Multipliers}    \makebox[16em][l]{交替方向乘子算法}
\end{abbreviationlist}

\XSYUpremainmatter